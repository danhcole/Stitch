\documentclass[11pt, oneside]{article}   	% use "amsart" instead of "article" for AMSLaTeX format
\usepackage{geometry}                		% See geometry.pdf to learn the layout options. There are lots.
\geometry{letterpaper}                   		% ... or a4paper or a5paper or ... 
%\geometry{landscape}                		% Activate for for rotated page geometry
\usepackage[parfill]{parskip}    			% Activate to begin paragraphs with an empty line rather than an indent
\usepackage{graphicx}				% Use pdf, png, jpg, or eps§ with pdflatex; use eps in DVI mode
								% TeX will automatically convert eps --> pdf in pdflatex	
\usepackage{amssymb}
\usepackage{amsmath}
\usepackage{listings}
\usepackage{color}

\date{}							% Activate to display a given date or no date

\newcommand{\tab} {\hspace*{2em}}
\newcommand{\stab} {\hspace*{.75em}}

\definecolor{codegreen}{rgb}{0,0.6,0}
\definecolor{codepurple}{rgb}{0.58,0,0.82}
\definecolor{codeblue}{rgb}{0,0,0.6}
\definecolor{codeorange}{rgb}{1,0.25,0}
\definecolor{backcolour}{rgb}{0.95,0.95,0.92}

\lstdefinestyle{customc}{
  backgroundcolor=\color{backcolour},   
  keywordstyle=\bfseries\color{codegreen},
  commentstyle=\itshape\color{codepurple},
  identifierstyle=\color{codeblue},
  stringstyle=\color{codeorange},
  basicstyle=\footnotesize,
  breakatwhitespace=false,         
  breaklines=true,                 
  captionpos=b,                    
  keepspaces=true,                 
  numbers=left,                    
  numbersep=5pt,                  
  showspaces=false,                
  showstringspaces=false,
  showtabs=false,                  
  tabsize=2
}

\lstset{escapechar=@,style=customc}

\begin{document}

\begin{center}
\LARGE
Stitch Language Proposal\\[2em]
\Large 
Daniel Cole, Megan Skrypek, Rashedul Haydar, Tim Waterman\\
\large dhc2131, ms4985, rh2712, tbw2105\\[2em]
\normalsize
September 30, 2015\\[3em]
\end{center}

\LARGE\textbf{Motivation}\\[.5em]
\normalsize
Most "modern" programming languages trace their origins back decades to before the advent of cheap, general purpose multicore CPUs.  They were designed for a distinctly mono-threaded environment.  While libraries and enhancements to mainstay languages such as C/C++ and Java have added multithreading capabilities, it remains in many ways bolted on kludge.  While newer frameworks such as Node.js provide more integral support for asynchronous operations, they lack the depth of support and power of a fully compiled language.  With Stitch, we aim to build a language that has the power and flexibility of a fully compiled C style language, while having native threading support for modern multithreaded applications.  Our goal is to create a translator from Stitch to C.
\\[3em]
\LARGE\textbf{Description}\\[.5em]
\normalsize
Stitch is inspired by C, which has a very well known syntax, and has been one of the most widely used languages since it was released over forty years ago.  Stitch is a general purpose language that supports all standard mathematical and logical operations.  Like C, Stitch is strongly typed, and whitespace does not matter.\\[.5em]
In addition to the standard C primitive types (\verb|int, double, char|, etc.), Stitch has native support for the \verb|string| type.  This includes concatenation, and a inbuilt length operator.  Stitch also has support for the \verb|boolean| type.\\[.5em]
What sets Stitch apart is its native support for multithreading using the \verb|async| keyword.  This keyword can be applied to any function call, as well as to any loop construct.  When called in this way, functions and loops will run in their own thread. 

\newpage

\LARGE\textbf{Examples}\\[.5em]
\normalsize
\lstinputlisting[language=C]
{Sink.txt}

\end{document}
